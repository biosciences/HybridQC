\section{Summary}\label{summary}

HybridQC is an R package that streamlines quality control (QC) of
single-cell RNA sequencing (scRNA-seq) data by combining traditional
threshold-based filtering with machine learning--based outlier
detection. It provides an efficient and adaptive framework to identify
low-quality cells in noisy or shallow-depth datasets using techniques
such as Isolation Forest, while remaining compatible with widely adopted
formats such as Seurat objects.

The package is lightweight, easy to install, and suitable for
small-to-medium scRNA-seq datasets in research settings. HybridQC is
especially useful for projects involving non-model organisms, rare
samples, or pilot studies, where automated and flexible QC is critical
for reproducibility and downstream analysis.

\section{Statement of Need}\label{statement-of-need}

scRNA-seq experiments often suffer from technical noise, dropout events,
and variability in sequencing depth. Traditional quality control relies
on static cutoffs for metrics such as gene count, UMI count, and
mitochondrial content, which may be suboptimal for non-standard
datasets.

HybridQC fills this gap by integrating machine learning
methods---specifically unsupervised outlier detection---with traditional
QC to improve filtering precision and robustness. This dual-level
approach can better preserve informative but unconventional cell types
and adapt dynamically to diverse datasets. No existing R packages
provide this hybrid QC strategy as a standalone tool with seamless
integration into Seurat-based pipelines.

\section{Features}\label{features}

\begin{itemize}
\tightlist
\item
  Computes standard QC metrics: \texttt{nFeature\_RNA},
  \texttt{nCount\_RNA}, \texttt{percent.mt}
\item
  Supports Isolation Forest outlier detection via \texttt{reticulate}
  and \texttt{pyod}
\item
  Filters cells using a hybrid decision rule
\item
  Works on Seurat objects
\item
  Lightweight and suitable for quick prototyping or small studies
\end{itemize}

\section{Example Usage}\label{example-usage}

\begin{Shaded}
\begin{Highlighting}[]
\FunctionTok{library}\NormalTok{(Seurat)}
\FunctionTok{library}\NormalTok{(HybridQC)}

\NormalTok{pbmc }\OtherTok{\textless{}{-}} \FunctionTok{LoadPBMC2k}\NormalTok{()  }\CommentTok{\# Load example data}
\NormalTok{qc\_basic }\OtherTok{\textless{}{-}} \FunctionTok{run\_basic\_qc}\NormalTok{(pbmc)}
\NormalTok{ml\_scores }\OtherTok{\textless{}{-}} \FunctionTok{run\_isolation\_forest\_qc}\NormalTok{(pbmc)}
\NormalTok{filtered }\OtherTok{\textless{}{-}} \FunctionTok{filter\_cells}\NormalTok{(pbmc, qc\_basic, ml\_scores)}
\end{Highlighting}
\end{Shaded}

\section{Acknowledgements}\label{acknowledgements}

The author thanks collaborators at University of Sydney for feedback on
early concepts.

\section{References}\label{references}

• Luecken, M. D., Büttner, M., Chaichoompu, K., Danese, A., Interlandi,
M., Mueller, M. F., Strobl, D. C., Zappia, L., Dugas, M., Colomé-Tatché,
M., \& Theis, F. J. (2022). Benchmarking atlas-level data integration in
single-cell genomics. Nature Methods, 19(1), 41--50.
https://doi.org/10.1038/s41592-021-01336-8 • McInnes, L., Healy, J., \&
Melville, J. (2018). UMAP: Uniform Manifold Approximation and Projection
for Dimension Reduction. arXiv preprint arXiv:1802.03426.
https://arxiv.org/abs/1802.03426 • Zhao, Y., Nasrullah, Z., \& Li, Z.
(2018). PyOD: A Python Toolbox for Scalable Outlier Detection. arXiv
preprint arXiv:1901.01588. https://arxiv.org/abs/1901.01588
